

%------------------------------------------------------------------------------------------------------------------------------------------------


%========================================================
%Macroproceso Gestion de Personal
%========================================================

%========================================================
% Descripción general del proceso
%-----------------------------------------------
\begin{Proceso}{P1.0}{Gestion de Personal} {
  
  %-------------------------------------------
  %Resumen
Macroproceso que lleva a cabo el \cdtRef{Actor:Jefe de Biblioteca}{Jefe de Biblioteca} para poder hacer la peticiónes para el registro,modificaciones y eliminaciones en el sistema de personal de la biblioteca. Para cada actividad hay un subproceso.
 


  %-------------------------------------------
  %Diagrama del proceso

  \noindent La Figura \cdtRefImg{P1.0}{Gestion de Personal} muestra las actividades que se realizan para llevar a cabo el proceso descrito anteriormente.

  \Pfig[0.95]{./procesos/C1_Empleados/Images/GU1_0-GestionPersonal.png}{P1.0}{Gestion de Personal}

} {P1.0:Gestion de Personal}

  %-------------------------------------------
  %Elementos del proceso

  \UCitem{Actores} { %Actores
    \cdtRef{Actor:Jefe de Biblioteca}{Jefe de Biblioteca}.
  }

  \UCitem{Objetivo} { %Objetivo
    Hacer el registro, la consulta, la actualizacion de datos y eliminacion del personal de la biblioteca.
  }

  \UCitem{Insumos de entrada} { %Insumos de entrada
	 Seleccion de la actividad a realizar.
  }
  
  \UCitem{Proveedores} { %Proveedores
  \begin{UClist}
  		\UCli Direccion de Capital Humano de ESCOM
    	\UCli Biblioteca General del IPN
    \end {UClist}
    
  }

  \UCitem{Productos de salida} { %Productos de salida
    \begin{UClist}
    
      \UCli Depende del subproceso a realizar.
    \end{UClist}
  }

  \UCitem{Cliente} { %Cliente
    \cdtRef{Actor:Jefe de Biblioteca}{Jefe de Biblioteca}
  }

  \UCitem{Mecanismo de medición} { %Mecanismo de medición
    \begin{UClist}
      \UCli Respuesta inmediata
      
    \end{UClist}
  }


\end{Proceso}

%========================================================
%Descripción de tareas
%-----------------------------------------------
\begin{PDescripcion}

  %Actor: Jefe de Bibliotecca
  \Ppaso Jefe de Biblioteca

    \begin{enumerate}

      %Tarea a
      \Ppaso[\itarea] \cdtLabelTask{T1-P1.0:Jefe de Biblioteca}{Registrar Usuario.} Subproceso en el cual el \cdtRef{Actor:Jefe de Biblioteca}{Jefe de Biblioteca} podra realizar el registro de nuevo personal a la biblioteca. 

\Ppaso[\itarea] \cdtLabelTask{T2-P1.0:Jefe de Biblioteca}{Modificar Usuario.} Subproceso en el cual el \cdtRef{Actor:Jefe de Biblioteca}{Jefe de Biblioteca} podra actualizar los datos del personal.

\Ppaso[\itarea] \cdtLabelTask{T3-P1.0:Jefe de Biblioteca}{Eliminar Usuario.} Subproceso en el cual el \cdtRef{Actor:Jefe de Biblioteca}{Jefe de Biblioteca} podra dar de baja de las funciones de la biblioteca al personal indicado.
	

    \end{enumerate}
    
    
\end{PDescripcion}


%------------------------------------------------------------------------------------------------------------------------------------------------
