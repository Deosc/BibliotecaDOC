

%------------------------------------------------------------------------------------------------------------------------------------------------


%========================================================
%Proceso Registrar Personal
%========================================================

%========================================================
% Descripción general del proceso
%-----------------------------------------------
\begin{Proceso}{P1.1}{Registrar Personal} {
  
  %-------------------------------------------
  %Resumen
Proceso que lleva a cabo el \cdtRef{Actor:Jefe de Biblioteca}{Jefe de Biblioteca} para poder hacer la petición y el registro de
nuevo personal.

De manera más especifica este proceso, realiza una petición a la biblioteca general solicitando más personal como pueden ser un nuevo \cdtRef{Actor:Bibliotecario}{Bibliotecario} o un nuevo \cdtRef{Actor:Personal de Procesos Tecnicos}{Personal de Procesos Tecnicos}.
La biblioteca general regresa una respuesta que puede ser afirmativa o negativa.

En caso de que la solicitud haya sido aceptada, la Dirección de capital humano envía los datos del
nuevo personal. Tales datos son el nombre completo, la cedula profesional y el rol a desempeñar
dentro de la biblioteca.
  


  %-------------------------------------------
  %Diagrama del proceso

  \noindent La Figura \cdtRefImg{P1.1}{Registrar Personal} muestra las actividades que se realizan para llevar a cabo el proceso descrito anteriormente.

  \Pfig[0.95]{./procesos/C1/Images/GU1_1-RegistrarPersonal.png}{P1.1}{Registrar Personal}

} {P1.1:Registrar Personal}

  %-------------------------------------------
  %Elementos del proceso

  \UCitem{Actores} { %Actores
    \cdtRef{Actor:Jefe de Biblioteca}{Jefe de Biblioteca}.
  }

  \UCitem{Objetivo} { %Objetivo
    Hacer el registro del nuevo personal, para que este tenga privilegios según sea su rol en la biblioteca.
  }

  \UCitem{Insumos de entrada} { %Insumos de entrada
  	\begin{UClist}
  		\UCli Datos del Formulario \cdtIdRef{F1.1}{Registrar Personal}.
    \end {UClist}
  }
  
  \UCitem{Proveedores} { %Proveedores
    Direccion de Capital Humano de ESCOM
  }

  \UCitem{Productos de salida} { %Productos de salida
    \begin{UClist}
    
      \UCli Tabla generada con los datos correspondientes del \cdtRef{Actor:Bibliotecario}{Bibliotecario} o \cdtRef{Actor:Personal de Procesos Tecnicos}{Personal de Procesos Tecnicos} y su Estado en el sistema.
      \UCli   Notificación \cdtIdRef{MSJ1.1}{Operación exitosa}.
    \end{UClist}
  }

  \UCitem{Cliente} { %Cliente
    \cdtRef{Actor:Jefe de Biblioteca}{Jefe de Biblioteca}
  }

  \UCitem{Mecanismo de medición} { %Mecanismo de medición
    \begin{UClist}
      \UCli Respuesta inmediata
      
    \end{UClist}
  }
  \UCitem{Interrelación con otros procesos} { %Interrelación con otros procesos
    \cdtIdRef{P1.2}{Modificar Personal}
}


\end{Proceso}

%========================================================
%Descripción de tareas
%-----------------------------------------------
\begin{PDescripcion}

  %Actor: Jefe de Bibliotecca
  \Ppaso Jefe de Biblioteca

    \begin{enumerate}

      %Tarea a
      \Ppaso[\itarea] \cdtLabelTask{T1-P1.1:Jefe de Biblioteca}{Solicitud de nuevo personal.}Un administrativo genera una solicitud de personal a biblioteca general, esta puede ser aceptada o rechazada. En caso de rechazo finaliza el proceso, por último si la solicitud es aceptada se puede proceder a la siguiente tarea. 

\Ppaso[\itarea] \cdtLabelTask{T2-P1.1:Jefe de Biblioteca}{Registro de personal.} El administrativo recibe un documento donde se especifican los datos del nuevo personal como son nombre completo, cedula profesional y finalmente se asigna el rol que desempeñaran dentro de la biblioteca.

	

    \end{enumerate}
    
    
\end{PDescripcion}


%------------------------------------------------------------------------------------------------------------------------------------------------
