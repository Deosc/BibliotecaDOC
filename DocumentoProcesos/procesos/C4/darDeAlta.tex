

%------------------------------------------------------------------------------------------------------------------------------------------------


%========================================================
%Proceso Cambiar estado de Alumno
%========================================================

%========================================================
% Descripción general del proceso
%-----------------------------------------------
\begin{Proceso}{P4.1}{Dar de Alta Lector} {
  
  %-------------------------------------------
  %Resumen

Proceso que realiza el \cdtRef{Actor:Lector}{Lector}, con el fin de tener acceso a los prestamos bibliotecarios, y tenga el derecho de estar monitoreado su perfil y así visualizar el \cdtRef{Actor:Bibliotecario}{Bibliotecario} en qué estado se encuentra el \cdtRef{Actor:Lector}{Lector}.
  
  
Para poder tener un Registro en la Biblioteca, el \cdtRef{Actor:Lector}{Lector} deberá de contar con los requisitos del formulario que pide dicho organismo para validar los datos, y que haga valer la  \cdtIdRef{C4.1}{Lector Inscrito en el instituto} y la \cdtIdRef{C4.3}{Formato llenado correctamente}, ya que al no estar inscrito a la escuela, no podrá formar parte del proceso de registro a la Biblioteca.
  


  %-------------------------------------------
  %Diagrama del proceso

  \noindent La Figura \cdtRefImg{P4.1}{Dar de Alta Lector} muestra las actividades que se realizan para llevar a cabo el proceso descrito anteriormente.

  \Pfig[0.95]{./procesos/C4/Images/GU4_1-DardeAltaLector.JPG}{P4.1}{Dar de Alta Lector}

} {P4.1:Dar de Alta Lector}

  %-------------------------------------------
  %Elementos del proceso

  \UCitem{Actores} { %Actores
    \cdtRef{Actor:Bibliotecario}{Bibliotecario} y \cdtRef{Actor:Lector}{Lector}.
  }

  \UCitem{Objetivo} { %Objetivo
    Validar Datos y generar el registro del \cdtRef{Actor:Lector}, para que este tenga privilegios según el perfil en la biblioteca.
  }

  \UCitem{Insumos de entrada} { %Insumos de entrada
  	\begin{UClist}
  		\UCli Datos del Formulario \cdtIdRef{F4.1}{Dar de alta Lector}.
    \end {UClist}
  }
  
  \UCitem{Proveedores} { %Proveedores
    \cdtRef{Actor:Lector}{Lector}
  }

  \UCitem{Productos de salida} { %Productos de salida
    \begin{UClist}
    
      \UCli Tabla generada con los datos correspondientes del  \cdtRef{Actor:Lector}{Lector} y su Estado en el sistema.
      \UCli   Notificación \cdtIdRef{MSJ4.1}{Lector no registrado}.
      \UCli   Notificación \cdtIdRef{MSJ4.5}{Operación exitosa}.
    \end{UClist}
  }

  \UCitem{Cliente} { %Cliente
    \cdtRef{Actor:Lector}{Lector}
  }

  \UCitem{Mecanismo de medición} { %Mecanismo de medición
    \begin{UClist}
      \UCli Respuesta inmediata
      
    \end{UClist}
  }
  \UCitem{Interrelación con otros procesos} { %Interrelación con otros procesos
    \cdtIdRef{P4.4}{Consultar historial de Lector}
  }


\end{Proceso}

%========================================================
%Descripción de tareas
%-----------------------------------------------
\begin{PDescripcion}

  %Actor: Aspirante
  \Ppaso Lector

    \begin{enumerate}

      %Tarea a
      \Ppaso[\itarea] \cdtLabelTask{T1-P4.1:Lector}{Inscribir al Instituto.}Tendra que inscribirse a la Escuela para que el Bibliotecario pueda generar el registro al organismo y proporcionar los datos correspondientes al \cdtRef{Actor:Bibliotecario}{Bibliotecario} para que este pueda validar la información adquirida y pueda empezar el registro a la Biblioteca. 

\Ppaso[\itarea] \cdtLabelTask{T2-P4.1:Lector}{Completa Datos.} Proporcionará los datos completos, según el Formato 1, de lo contrario termina el proceso.

	

    \end{enumerate}
    
    
      %Actor: SAEV2.0
  \Ppaso Bibliotecario 

    \begin{enumerate}

      %Tarea a
      \Ppaso[\itarea] \cdtLabelTask{T1-P4.1:Bibliotecario}{Validar datos de Lector.} Valida los datos del \cdtRef{Actor:Lector}{Lector} Según el Formato 1 que estén completos y que sea un \cdtRef{Actor:Lector}{Lector} inscrito y/o Registrado a la escuela, si la regla de negocio se cumple pasa a la tarea \cdtRefTask{T2-P4.1:Bibliotecario}{Datos Incompletos.}, donde valida por completo que los datos estén completos, si no cumple con la regla de negocio, termina el proceso. Este notifica al \cdtRef{Actor:Lector}{Lector} que deberá de inscribirse a la escuela.

%referenciar tareas \cdtRefTask{T2-P0.1:SAEV2.0}{Notifica solicitud fuera de periodo de registro.}


      %Tarea b
      \Ppaso[\itarea] \cdtLabelTask{T2-P4.1:Bibliotecario}{Datos Incompletos.} Valida que los datos este completos, si los datos están completos, pasa a la tarea \cdtRefTask{T3-P4.1:Bibliotecario}{Registro.} que registra los datos en el sistema, si no están completos el sistema genera una notificación que mostrara al Bibliotecario que los datos son incompletos, y este le notificara al Lector.
      
      %Tarea c
      \Ppaso[\itarea] \cdtLabelTask{T3-P4.1:Bibliotecario}{Registro.}Registra los datos del Lector en el sistema y genera la Notificación \cdtIdRef{MSJ4.5}{Operación exitosa} para mostrarlo al \cdtRef{Actor:Lector}{Lector}.
     

          \end{enumerate}

\end{PDescripcion}


%------------------------------------------------------------------------------------------------------------------------------------------------
