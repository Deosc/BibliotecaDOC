
\begin{BussinesRule}{RN4.1}{Campos no nulos.} 
	%\BRitem[Autor:] Miguel Ángel Castañeda Sánchez.
	\BRitem[Descripción:] Ningún dato en el formulario del lector puede ser nulo.
	\BRitem[Tipo:] Restricción (validación).
	\BRitem[Nivel:] Obligatorio.
\end{BussinesRule}

%---------------------------------------------------------

\begin{BussinesRule}{RN4.2}{Formato del número de boleta del IPN.}
	%\BRitem[Autor:] Miguel Ángel Castañeda Sánchez
	\BRitem[Descripción:] La boleta está compuesta por:
		\begin{itemize} 
			\item Año de ingreso del estudiante(4 dígitos).
			\item Número de matrícula de la escuela(2 dígitos).
			\item Número de estudiante(4 dígitos).
		\end{itemize}
	Ejemplo:\\
		Año de ingreso: 2014\\
		Escom: 63\\
		No. Estudiante: 147\\
		Su boleta seŕa: 2014630147\\
	\BRitem[Tipo:] Restricción (validación).
	\BRitem[Nivel:] Obligatorio.
\end{BussinesRule}

%---------------------------------------------------------

\begin{BussinesRule}{RN4.3}{Formato del nombre.}
	%\BRitem[Autor:] Miguel Ángel Castañeda Sánchez.
	\BRitem[Descripción:] El nombre esta compuesto por:
		\begin{itemize} 
			\item Nombre
			\item Primer apellido
			\item Segundo apellido 
		\end{itemize}
		Todo el nombre debe estar compuestos por letras.\\\\
Ejemplo \\
	Nombre: Luis Ángel, Apellido Paterno: Martínez, Apellido Materno: Gómez.
	\BRitem[Tipo:] Restricción(validación)
	\BRitem[Nivel:] Obligatorio.
\end{BussinesRule}

%---------------------------------------------------------

\begin{BussinesRule}{RN4.4}{Formato del CURP.}
	%\BRitem[Autor:] Miguel Ángel Castañeda Sánchez.
	\BRitem[Descripción:] El formato del CURP esta compuesto por::
		\begin{itemize}
			\item Primera letra y la primera vocal del primer apellido,
			\item Primera letra del segundo apellido,
			\item Primera letra del nombre,
			\item Fecha de nacimiento sin espacios en orden de año, mes y dia; ejemplo 940608 (08 de Junio de 1994),
			\item letra del sexo (H o M);
			\item Dos letras correspondientes a la entidad de nacimiento;
			\item Primera consonante interna (no inicial) del primer apellido;
			\item Primera consonante interna (no inicial) del segundo apellido;
			\item Primera consonante interna (no inicial) del nombre,
			\item Dígito del 0-9 para fechas de nacimiento hasta el año 1999 y A-Z para fechas de nacimiento a partir del 2000,
			\item Dígito, para evitar duplicaciones.			
		\end{itemize}

Ejemplo:\\
	Nombre: Luis Ángel Sánchez Fernańdez, Sexo: Masculino, Fecha de nacimiento: 05 de Julio de 1994 y Estado: Colima.\\
	Su CURP será: SAFL940705HCMNRS09.

	\BRitem[Tipo:] Restricción(validación)
	\BRitem[Nivel:] Obligatorio.
\end{BussinesRule}

%---------------------------------------------------------

\begin{BussinesRule}{RN4.5}{Formato de la fecha.}
	%\BRitem[Autor:] Miguel Ángel Castañeda Sánchez.
	\BRitem[Descripción:] La fecha tiene el siguiente formato :
		\begin{itemize} 
			\item Dia(DD)/ 
			\item Mes(MM)/
			\item Año (AAAA)
		\end{itemize}
Ejemplo \\
	16/05/2017
	\BRitem[Tipo:] Restricción(validación)
	\BRitem[Nivel:] Obligatorio.
\end{BussinesRule}

%---------------------------------------------------------

\begin{BussinesRule}{RN4.6}{Formato de la dirección.} 
	%\BRitem[Autor:] Miguel Ángel Castañeda Sánchez.
	\BRitem[Descripción:]El formato de la dirección debe ser:
		\begin{itemize}
			\item Tipo y nombre de la vialidad, 
			\item Número del domicilio, 
			\item Colonia, 
			\item Código postal, 
			\item Municipio, 
			\item Entidad federativa.
		\end{itemize}
	Ejemplo: Av. Rosales, No.5217, Col. Panamericana, C.P. 07770, Naucalpan de Juárez, Puebla.	
	\BRitem[Tipo:] Restricción (validación).
	\BRitem[Nivel:] Obligatorio.
\end{BussinesRule}

%---------------------------------------------------------

\begin{BussinesRule}{RN4.7}{Formato de teléfono.} 
	%\BRitem[Autor:] Miguel Ángel Castañeda Sánchez.
	\BRitem[Descripción:] El teléfono debe de estar formado solamente con números.
	\BRitem[Tipo:] Restricción (validación).
	\BRitem[Nivel:] Obligatorio.
\end{BussinesRule}

%---------------------------------------------------------

\begin{BussinesRule}{RN4.8}{Formato del semestre.} 
	%\BRitem[Autor:] Miguel Ángel Castañeda Sánchez.
	\BRitem[Descripción:] El formato del semestre será elegido desde un menú en el cual se podra elegir la opción que va desde:
		\begin{itemize}
			\item 1er semestre
			\item 2do semestre
			\item .
			\item ..
			\item ...
			\item 10mo semestre en adelante
		\end{itemize}
	\BRitem[Tipo:] Restricción (validación).
	\BRitem[Nivel:] Obligatorio.
\end{BussinesRule}

%---------------------------------------------------------

\begin{BussinesRule}{RN4.9}{Formato del email.} 
	%\BRitem[Autor:] Miguel Ángel Castañeda Sánchez.
	\BRitem[Descripción:] El formato del email puede estar formado por letras, numeros.\\
Ejemplo:\\
	luis-calles@hotmail.com\\
	xxxxx@xxxx.com\\

	\BRitem[Tipo:] Restricción (validación).
	\BRitem[Nivel:] Obligatorio.
\end{BussinesRule}

%---------------------------------------------------------

\begin{BussinesRule}{RN4.10}{Formato de la contraseña.} 
	%\BRitem[Autor:] Miguel Ángel Castañeda Sánchez.
	\BRitem[Descripción:] La contraseña debe de tener un tamaño mínimo de 8 caracteres y un máximo de 16 caracteres, la cual está compuesta por:
		\begin{itemize}	
			\item Letras mayúsculas.
			\item Letras minúsculas.
			\item Dígitos.
			\item Caracteres no alfanúmericos, es decir, caracteres especiales.
		\end{itemize}
	\BRitem[Tipo:] Restricción (validación).
	\BRitem[Nivel:] Obligatorio.
\end{BussinesRule}

%---------------------------------------------------------

\begin{BussinesRule}{RN4.11}{Formato de la credencial.} 
	%\BRitem[Autor:] Miguel Ángel Castañeda Sánchez.
	\BRitem[Descripción:] El formato de la credencial debe de contener los siguientes datos:
		\begin{itemize}	
			\item ID del lector.
			\item Nombre completo del lector.
		\end{itemize}
	\BRitem[Tipo:] Restricción (validación).
	\BRitem[Nivel:] Obligatorio.
\end{BussinesRule}

%---------------------------------------------------------

\begin{BussinesRule}{RN4.12}{Formato del número de empleado del IPN.}
	%\BRitem[Autor:] Miguel Ángel Castañeda Sánchez
	\BRitem[Descripción:] El número del empleado esta compuesto por:
		\begin{itemize} 
			\item Año de ingreso del docente(4 dígitos).
			\item Dos dígitos para saber de donde es egresado(01 egresado del IPN, 10 egresado de otra universidad).
			\item Número de trabajador(4 dígitos).
		\end{itemize}
	Ejemplo:\\
		Año de ingreso: 2014\\
		Egersado de ESCOM: 01\\
		No. trabajador: 3147\\
		Su número de trabajador: 201403147\\
	\BRitem[Tipo:] Restricción (validación).
	\BRitem[Nivel:] Obligatorio.
\end{BussinesRule}

%---------------------------------------------------------

\begin{BussinesRule}{RN4.13}{Formato del departamento.} 
	%\BRitem[Autor:] Miguel Ángel Castañeda Sánchez.
	\BRitem[Descripción:] El formato del departamento será elegido desde un menú en el cual se podra elegir la opción que va desde:
		\begin{itemize}
			\item Formación básica
			\item Ciencias e ingeniería de la computación
			\item Ingeniería en sistemas computacionales
			\item Formación integral e institucional
		\end{itemize}
	\BRitem[Tipo:] Restricción (validación).
	\BRitem[Nivel:] Obligatorio.
\end{BussinesRule}

%---------------------------------------------------------

\begin{BussinesRule}{RN4.14}{Lector no registrado en el sistema} 
	%\BRitem[Autor:] Miguel Ángel Castañeda Sánchez.
	\BRitem[Descripción:] Verificar que el lector no este dado de alta el sistema. Prop+osito evitar duplicidad de lectores en el sistema.
	\BRitem[Tipo:] Restricción (validación).
	\BRitem[Nivel:] Obligatorio.
\end{BussinesRule}

%---------------------------------------------------------

\begin{BussinesRule}{RN4.15}{Lector no registrado en el sistema} 
	%\BRitem[Autor:] Miguel Ángel Castañeda Sánchez.
	\BRitem[Descripción:] Verificar que el lector no este dado de alta el sistema. Propósito evitar duplicidad de lectores en el sistema.
	\BRitem[Tipo:] Restricción (validación).
	\BRitem[Nivel:] Obligatorio.
\end{BussinesRule}

%---------------------------------------------------------

\begin{BussinesRule}{RN4.16}{Alumno vigente en el instituto} 
	%\BRitem[Autor:] Miguel Ángel Castañeda Sánchez.
	\BRitem[Descripción:] Verificar que el alumno este inscrito en el instituto en el semestre actual.
	\BRitem[Tipo:] Restricción (validación).
	\BRitem[Nivel:] Obligatorio.
\end{BussinesRule}

%---------------------------------------------------------

\begin{BussinesRule}{RN4.17}{Registro de alta del lector llenado correctamente } 
	%\BRitem[Autor:] Miguel Ángel Castañeda Sánchez.
	\BRitem[Descripción:] Verificar que el registro del lector sea llenado correctamente. Que la información ingresada en el formulario corresponda a los formatos de cada campo.
	\BRitem[Tipo:] Restricción (validación).
	\BRitem[Nivel:] Obligatorio.
\end{BussinesRule}

%---------------------------------------------------------

\begin{BussinesRule}{RN4.18}{Registro de alta del lector llenado correctamente } 
	%\BRitem[Autor:] Miguel Ángel Castañeda Sánchez.
	\BRitem[Descripción:] Verificar que el registro del lector sea llenado correctamente. Que la información ingresada en el formulario corresponda a los formatos de cada campo.
	\BRitem[Tipo:] Restricción (validación).
	\BRitem[Nivel:] Obligatorio.
\end{BussinesRule}

%---------------------------------------------------------

\begin{BussinesRule}{RN4.19}{Comprobar identidad del lector } 
	%\BRitem[Autor:] Miguel Ángel Castañeda Sánchez.
	\BRitem[Descripción:] Verificar que la identificacion del lector corresponda con los datos del perfil del lector.
	\BRitem[Tipo:] Restricción (validación).
	\BRitem[Nivel:] Obligatorio.
\end{BussinesRule}

%---------------------------------------------------------

\begin{BussinesRule}{RN4.20}{El lector tiene multas} 
	%\BRitem[Autor:] Miguel Ángel Castañeda Sánchez.
	\BRitem[Descripción:] El lector no puede actualizar sus datos hasta que no pague las multas pendientes que tenga.
	\BRitem[Tipo:] Restricción (validación).
	\BRitem[Nivel:] Obligatorio.
\end{BussinesRule}

%---------------------------------------------------------

\begin{BussinesRule}{RN4.21}{El lector tiene adeudos de material} 
	%\BRitem[Autor:] Miguel Ángel Castañeda Sánchez.
	\BRitem[Descripción:] El lector no puede actualizar sus datos hasta que no entregue el material adeudado que tenga. Con adeudado se refiere a material entregado que no ha sido entregado en su fecha de entrega.
	\BRitem[Tipo:] Restricción (validación).
	\BRitem[Nivel:] Obligatorio.
\end{BussinesRule}

%---------------------------------------------------------

\begin{BussinesRule}{RN4.22}{Lector dado de alta} 
	%\BRitem[Autor:] Miguel Ángel Castañeda Sánchez.
	\BRitem[Descripción:] El lector esta dado de alta en el sistema.
	\BRitem[Tipo:] Restricción (validación).
	\BRitem[Nivel:] Obligatorio.
\end{BussinesRule}

%---------------------------------------------------------

\begin{BussinesRule}{RN4.23}{Lector sin credencial.}
	\BRitem[Tipo:] Restricción (validación).
	%\BRitem[Autor:] Miguel Ángel Castañeda Sánchez.
	\BRitem[Descripción:] El lector que extravio su credencial, puede renovar su crendecial.
\end{BussinesRule}

%---------------------------------------------------------

